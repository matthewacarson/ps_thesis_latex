\documentclass[12pt]{article}

%%%% Load Packages %%%%
\usepackage[utf8]{inputenc}
\usepackage[english]{babel}
\usepackage{color}
\usepackage{url}
\usepackage{hyperref}
\usepackage[citestyle=authoryear, bibstyle=authoryear, sorting=nyt]{biblatex}
\usepackage{caption}
\usepackage{etoolbox}
\usepackage{fancyhdr}
\usepackage[margin=1in]{geometry}
\usepackage{graphicx}
\usepackage{parskip}
\usepackage{setspace}
\usepackage{titlesec}
\usepackage{booktabs}
\usepackage{multirow}
\usepackage[autostyle, english = american]{csquotes}
\MakeOuterQuote{"}
\AtEveryBibitem{\ifentrytype{book}{\clearfield{isbn}}{}}
\addbibresource{ps_honors_thesis.bib}
\usepackage[acronym, nomain, nopostdot]{glossaries}
\makeglossaries

\titleformat{\section}{\fontsize{12}{14}\bfseries\centering}{\thesection}{0.5em}{}

%%%% List of Acronyms and abbreviations %%%%

\newacronym{acs}{ACS}{American Community Survey}
\newacronym{udp}{UDP}{Urban Displacement Project}

%%%% Head height %%%%
\setlength{\headheight}{15pt}
%%%% Line spacing %%%%
\setstretch{2}
%%%% Paragraph spacing %%%%
\setlength{\parskip}{0pt}
%%%% Define indentation length %%%%
\newlength{\myindent}
\setlength{\myindent}{2em}
%%%% Set the hanging indent %%%%
\setlength{\bibhang}{\myindent}
%%%% Redefine the citation command to use a colon instead of a comma and pp. %%%%
\DeclareFieldFormat{postnote}{#1}
\DeclareFieldFormat{multipostnote}{#1}
%%%% Use colon (APSA Style) in-text citations (author year, page) %%%%
\renewcommand*{\postnotedelim}{\addcomma\space}
%%%% Set global text alignment to ragged-right %%%%
%\raggedright
%%%% Paragraph indentation %%%%
\setlength{\parindent}{\myindent}

%%%% Font %%%%
% \setmainfont{Times New Roman}

%%%% Define a variable for the title %%%%
%\newcommand{\myTitle}{CONSTRUCTION UNION\dots{HISTORICAL-COMPARATIVE}}
\newcommand{\imageWidth}{0.8\textwidth}

%%%% Page style %%%%
\pagestyle{fancy}
\fancyhf{} % clear all header and footer fields
%\fancyhead[R]{\thepage} % page number on the right side
\fancyhead[R]{\hyperlink{toc}{\thepage}} % page number on the right side, linked to the TOC
%\fancyhead[L]{\small \myTitle}
\renewcommand{\headrulewidth}{1pt} % header rule

% Customize abstract page
\renewenvironment{abstract}
  {\par\noindent\centering\textbf{Abstract}\par}
  {\noindent\raggedright}
%  {\par}

% Redefine the quote environment
\renewenvironment{quote}
  {\list{}{\leftmargin=\parindent\rightmargin=0pt}%
   \item\relax}
  {\endlist}
  
% Redefine the quote environment to make it single-spaced 
% and remove vertical space before and add one at the end
\AtBeginEnvironment{quote}{\singlespacing\setlength{\topsep}{0pt}\setlength{\partopsep}{0pt}}
\AtEndEnvironment{quote}{\vspace{0.5\baselineskip}}

\begin{document}
\setstretch{1.25}
\begin{titlepage}
  \thispagestyle{fancy}
  \pagenumbering{gobble}
%  \fancyhead[L]{Running head: \myTitle}
  \renewcommand{\headrulewidth}{0pt} % header rule
  \centering
  \vspace*{2in}
  The Policing of the ``Reserve Army'':\par
  Economic Inequality and Police Killings\par
  \vspace{1.2in}
  {Matthew A. Carson\par}
  \vspace{12pt}
  Department of Political Science\par
  University of California, Los Angeles\par
  \vspace{0.5in}
  {March 22, 2024\par}
%  \vfill
%  \wordcount
\end{titlepage}

% Blank page so that the abstract does not begin on the back of the cover page.
\thispagestyle{empty} % Remove header and footer

\vspace*{\fill}
\hspace*{\fill}
\begin{center}
    \noindent{}This page was intentionally left blank.
\end{center}
\hspace*{\fill}
\vspace*{\fill}

\clearpage

% Set page numbering to Roman for preliminary pages
\pagenumbering{roman}
\setstretch{2}
%\titlespacing*{\abstract}{0pt}{0pt}{0pt}
% Add abstract page
\begin{abstract}
This study examines the relationship between race, class, gentrification, and police use of lethal force (LUOF) in U.S. census tracts. Analyzing US Census data from 2015 to 2020 reveals that while there is a disproportionate incidence of LUOFs in the majority non-white census tracts and of non-white victims, rates within racial groups differ significantly by income. Across all racial/ethnic groups, lower-income census tracts experience more LUOFs than higher-income tracts, but these income-based disparities are sharpest within majority black and majority Latino tracts. These findings suggest that while class matters across all groups, for blacks and Latinos, the class disparity is even greater. Gentrification’s impact on LUOF rates is more nuanced. In general, gentrifying tracts did not experience a greater LUOF rate than low-income, nongentrifying tracts. However, within majority black census tracts, those undergoing gentrification experienced the highest LUOF rate.
\end{abstract}

\clearpage
\setstretch{1.25}
% Add table of contents
\hypertarget{toc}{}
\tableofcontents
\clearpage

% List of acronyms
%\addcontentsline{toc}{section}{Acronyms}
\printglossary[type=\acronymtype,title=Acronyms]

% List of Figures
\listoffigures

% List of tables
\listoftables
\clearpage

\setstretch{2}

% Define custom section headings
\titleformat{\part}[block]{\normalfont\Large\bfseries\MakeUppercase}{\partname\ \thepart:}{6pt}{\Large\MakeUppercase}
\titleformat{\section}[block]{\normalfont\fontsize{12}{14}\selectfont\bfseries}{\thesection.}{0.25em}{}
\titleformat{\subsection}[block]{\itshape\bfseries}{\thesubsection.}{0.25em}{}
\titleformat{\subsubsection}[runin]{\normalfont\itshape\bfseries}{\hspace{\myindent}\thesubsubsection.}{0.25em}{}[.]%{\hspace{0em}}%[\hspace{2pt}]

% Adjust spacing before and after sectioning commands
\titlespacing*{\part}{0pt}{0pt}{10pt}
\titlespacing*{\section}{0pt}{0pt}{0pt} % \baselineskip
\titlespacing*{\subsection}{0pt}{0pt}{0pt}
\titlespacing*{\subsubsection}{0pt}{0pt}{0.35em}


% Set page numbering to Arabic for main content
\pagenumbering{arabic}

\part{INTRODUCTION} \

On August 8, 2015, US senator and Democratic presidential candidate Bernie Sanders spoke at “Social Security Works,” an event commemorating the 50th and 80th anniversary of the enactment of Social Security and Medicare, respectively (\cite{wilsonProtestersShutBernie2015}). But before Senator Sanders could speak, several Black Lives Matter activists interrupted because they felt Sanders was inadequately responding to issues of racial justice, particularly as it pertains to the killings of black Americans by law enforcement. One activist, Marissa Johnson, in an MSNBC interview, elaborated that the interruption aimed to “put pressure on people who claim that they care about black lives” (\cite{hallBernieSandersBlack2015}). Specifically, regarding Sanders, she averred, “if you look at Bernie Sanders’s platform, you look at what he said on racial equality, he’s basically a class reductionist. He’s never really had a strong analysis that there is racism and white supremacy that is separate than [\textit{sic}] the economic things that everyone experiences. So, we want to continue to push him on that” (\cite{hallBernieSandersBlack2015}). The issue for Johnson and the other activists who stormed the stage that day was that while Sanders had a fairly expansive economic justice platform, those policies were an inadequate response to issues of racism and white supremacy. Hence, from this perspective, his politics were class reductionist.

Earlier that year, in an interview with CNN’s Wolf Blitzer, Senator Sanders was asked about his thoughts on the unrest in Baltimore following the murder of Freddie Gray by law enforcement. Sanders emphasized that “too many mostly black suspects have been treated terribly and, in some cases, murdered,” and that “police officers have got to be held accountable for their actions,” but also that economic factors were related to the killing of Freddie Gray:

\begin{quote}
[I]n the neighborhood where this gentleman [Freddie Gray] lives [\textit{sic}], as I understand it, the unemployment rate is over 50 percent, over 50 percent. What we have got to do as a nation is understand that we have got to create millions of jobs to put people back to work to make sure that kids are in schools and not in jails. So, short term, we've got to make sure that police officers have cameras. We've got to make sure that we have real police reform so that suspects are treated with respect. Long term, we've got to make sure that our young people are working, they're in school, they're not hanging out on street corners. (\cite{sandersInterviewWolfBlitzer2015})
\end{quote}

That is, for Sanders, people being killed by law enforcement is inextricably tied with
unemployment and economic inequality: Those residents in Freddie Gray’s neighborhood had
little economic opportunity, which meant that they would frequently “hang out on street corners”
and come into contact with police, often with deadly consequences. This was in contradistinction
to the claims of activists like Marissa Johnson, who saw the issue primarily as a function of
racism and white supremacy.

This project aims to take on this question. Of course, racial discrimination and economic inequality are not mutually exclusive, and this project does not suggest such a view. However, these two contrasting perspectives vis-à-vis police violence are worth exploring further. As I will contend, while African Americans are indeed disproportionately targeted and killed by law enforcement, the phenomenon is also much broader and affects many low-income people more generally. As researchers who are attempting to better understand the phenomenon, it is imperative that we not only understand the racial disparity frame of reference but also the role of economic inequality.

\section{Background}\

From a disciplinary perspective, Political Science has not adequately researched issues of policing and incarceration (a related phenomenon). In their 2017 article, “Police Are Our Government: Politics, Political Science, and the Policing of Race–Class Subjugated Communities,” in the \textit{Annual Review of Political Science}, Joe Soss and Velsa Weaver highlighted how the discipline has failed to “heed the call” for greater research into the issues concerning policing (\citeyear[568]{sossPoliceAreOur2017}). Consequently, the discipline “continues to offer a distorted portrait of democracy and government in America and a deeply incomplete view of how politics and power operate in RCS [race and class subjugated] communities” (\cite[568]{sossPoliceAreOur2017}). Significantly, the authors call upon the discipline to more closely examine the state’s second face: “the activities of governing institutions and officials that exercise social control and encompass various modes of coercion, containment, repression, surveillance, regulation, predation, discipline, and violence” (\cite[567]{sossPoliceAreOur2017}). It is in this spirit that this paper proceeds, as an undertaking aimed at better understanding these dynamics.

The rates of lethal uses of police force are remarkably high in the United States relative to other countries in the Global North, making it all the more urgent of an issue. While this project is not chiefly focused on comparative aspects, it nonetheless helps drive home the point regarding how serious of an issue this is. \citeauthor{espinerLicenceKillStartling2022} (\citeyear{espinerLicenceKillStartling2022}) observed that “America is in a league of its own with nearly 31 police shootings per 10 million people,” making the United States’s rate nearly four times that of New Zealand, and over 100 times that of England and Wales. Other countries that \citeauthor{espinerLicenceKillStartling2022} (\citeyear{espinerLicenceKillStartling2022}) investigated include Canada (9.2 per ten million) and Norway (3.6 per ten million). This underscores the urgency and need to further investigate causal forces contributing to the high incidence in the United States.

\subsection{Research Question}\

Are lower-income people more likely to be killed by law enforcement, even when considering other variables such as race?

\part{DATA AND METHODS}\

\section{Data}\

Three sources will be used in the project. The Fatal Encounters data set is the primary source of incidents of someone dying in the course of police activity. Journalist D. Brian Burghart started the effort in 2012 after finding that there was no comprehensive database of people killed during interactions with the police. Data have been collected using paid researchers who aggregate data from other large data sets such as the \textit{Los Angeles Times}’ “Homicide Report,” public records requests, and crowdsourced data (\cite{burghartMeFatalEncounters}). Crowd-sourced data is subsequently checked against published media reports or public records to verify accuracy. Every incident includes a link to a public record or media report substantiating the veracity of the details of the death (\cite{burghartMeFatalEncounters}). Because of the limitations of the FBI’s Uniform Crime Report (i.e., participation by law enforcement agencies is voluntary, and the number of persons killed by law enforcement is severely underreported), Fatal Encounters is one of the main sources that academics use when researching police use of deadly force (\cite{feldmanKilledPoliceValidity2017, feldmanQuantifyingUnderreportingLawenforcementrelated2017, feldmanPoliceRelatedDeathsNeighborhood2019}).

The US Census’ \acrfull{acs} is the source of median household income data for each census tract. Since each incident in the Fatal Encounters data set includes a latitude and longitude, it can be matched with a census tract in the \acrshort{acs}. Census tracts are “small, relatively permanent statistical subdivisions of a county or statistically equivalent entity” that “generally have a population size between 1,200 and 8,000 people, with an optimum size of 4,000 people” (\cite{bureauGlossary}).

The \acrfull{udp} is the source of gentrification data. The \acrshort{udp} conducts “data-driven, applied research,” including census tract-level identification of gentrification or lack thereof (\cite{udpDisplacementGentrificationTypologies2023}). The UDP has constructed nine typologies based on income and Zillow home values and changes in income or Zillow home values between 2000 and 2018. Because of how difficult it would be to interpret the relationship between such a large number of typologies and police lethal use of force, several of these typologies have been merged




\part{CONCLUSION}\

{\color{red}[Conclusion goes here]}

\clearpage

\titleformat{\section}{\fontsize{12}{14}\bfseries\centering}{\thesection}{0.5em}{}
% Line spacing
\setstretch{1.25}
\printbibliography[heading=bibintoc]

\clearpage

\paperwidth=\pdfpageheight
\paperheight=\pdfpagewidth
\pdfpageheight=\paperheight
\pdfpagewidth=\paperwidth
\headwidth=\textheight
\begingroup 
\vsize=\textwidth
\hsize=\textheight

%\newpage
%\paperwidth=\pdfpageheight
%\paperheight=\pdfpagewidth
%\pdfpageheight=\paperheight
%\pdfpagewidth=\paperwidth
%\headwidth=\textwidth

\end{document}
